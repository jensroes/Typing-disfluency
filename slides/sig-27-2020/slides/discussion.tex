% summary of claim
\begin{frame}{Summary}
	\begin{itemize}
		\item Better fit for mixture models over the standard analysis.
		\item Sentences starting with conjoined NPs are associated with a larger proportion of long latencies.
		\item This proportion was relatively small; too small for an obligatory (default) planning process \parencite{martin2010planning}.
		\item Against expectation, similar results, of a smaller magnitude, were found when the syntactic component was removed.
	\end{itemize}
	
\end{frame}


\begin{frame}{Summary}
	\begin{itemize}
		\item No evidence for phrase-as-planning-unit hypothesis: NP syntax isn't obligated by production system.
		\item Instead, the slowdown for conjoined NPs is better explained by a larger probability of long latencies which, however, remained in a minority.
		\item Syntax in language production must result from a non-deterministic planning mechanism.
		\item and is \textbf{not} entirely independent of the visual manipulation.
	\end{itemize}
\end{frame}


% Better fit for MoG
% Larger prop for conjoined
% still too small for an obligatory (default) process
% Very small but "reliable" effect for stim manipulation due to larger sample size?
% Larger prop for conjoined
% -> but prop overall smaller
% -> RT overall shorter

\begin{comment}
\begin{frame}{Alternative explanations}
	\begin{itemize}
%		\item Results are not a by-product of the visual manipulation. %observed in a subset of trials origins during visual rather than grammatical encoding. 
%		\item Both a relational and non-relational route are available during high level encoding \parencite{kuchinsky2011reversing,konopka2014priming}. 
		\item[i.] Lexical preplanning: avoidance of intra-sentential pausing.
		\item[ii.] Syntactic correction of incorrectly activated NP syntax.
		\item[iii.] Both: use of syntactic route, instead of a lexical route, leads to a slowdown in conjoined NPs but not in simple NPs. 
	\end{itemize}
\end{frame}
\end{comment}
