\begin{frame}{Conclusion}
	\begin{itemize}
		\item Better fit for mixture models over standard analysis.
		\item Capture writing process as a mixture of fluent and disfluent key transitions.
		\item Advantages of mixture models for writing research:
		\begin{enumerate}
			\item map on cascading models of writing.
			\item capture disfluencies in a principled way.
			\item represent the probabilistic nature of disfluencies.
			\item provide reliable typing estimates and pause frequencies.
			
		\end{enumerate}
	\end{itemize}
\end{frame}


\begin{comment}
\begin{frame}{Discussion}
	\begin{itemize}
		\item Our statistical techniques need to align closely with the cognitive process we're trying to understand.
		\item \dots and represent our current understanding of the underlying cognitive process.
		\item To achieve this we need to model the raw data possible rather than summary statistics.
		\item Otherwise we risk incorrect conclusions about our data.
	\end{itemize}	
\end{frame}

\begin{frame}{Prospects}	
	\begin{itemize}
		\item Shiny-app: calculate disfluencies from keystroke data.
		\item Manuscript: extension to autoregression.
		\item Application of models for data in which pause frequencies play a central role: group comparisons, diagnostic tool?
		\item Application picture description data with spelling manipulation.
	\end{itemize}
\end{frame}
\end{comment}