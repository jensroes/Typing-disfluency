\begin{frame}{Method}

\begin{backgroundblock}{5mm}{15mm}
	\begin{tikzpicture}[framed,background rectangle/.style={thick, rounded corners, draw=black}]
		\node[inner sep=1pt] (screen) at (10,-5)
    		{\includegraphics[scale=.55]{gfx/method/build/screen-3}};   		
    		\node[font = \scriptsize, below = 0cm of screen] {Stimulus array: A and B moved above C};
	\end{tikzpicture}
\end{backgroundblock}

\begin{backgroundblock}{65mm}{15mm}
	\begin{tikzpicture}[framed,background rectangle/.style={thick, rounded corners, draw=black}]
		\node[inner sep=3.5pt, visible on=<2->] (screen) at (10,-5)
    		{\includegraphics[scale=.55]{gfx/method/build/screen-4}};   			    					\node[font = \scriptsize, below = 0cm of screen, visible on=<2->] {Stimulus array: A moved above B and C};
	\end{tikzpicture}
\end{backgroundblock}
\vspace{3cm}
	\begin{itemize}
		\item \uncover<3->{48 items; 96 fillers; 6 practice trials}
		\item \uncover<4->{First noun: \textit{Peter} or \textit{Tania}}
		%	\item \uncover<4>{Movement: up or down}
		\item \uncover<5->{Image names: high frequency and naming agreement.}
	\end{itemize}

\end{frame}

\begin{frame}{Experiment 1}
\begin{backgroundblock}{5mm}{15mm}
	\begin{tikzpicture}[framed,background rectangle/.style={thick, rounded corners, draw=black}]
		\node[inner sep=1pt] (screen) at (10,-5)
		{\includegraphics[scale=.55]{gfx/method/build/screen-3}};   		
		\node[font = \scriptsize, below = 0cm of screen] {Stimulus array: A and B moved above C};
	\end{tikzpicture}
\end{backgroundblock}

\begin{backgroundblock}{65mm}{15mm}
	\begin{tikzpicture}[framed,background rectangle/.style={thick, rounded corners, draw=black}]
		\node[inner sep=3.5pt, visible on=<2->] (screen) at (10,-5)
		{\includegraphics[scale=.55]{gfx/method/build/screen-4}};   			    					\node[font = \scriptsize, below = 0cm of screen, visible on=<2->] {Stimulus array: A moved above B and C};
	\end{tikzpicture}
\end{backgroundblock}

\vspace{3cm}

\begin{itemize}
	\item \uncover<1->{``\textbf{Peter and the dog} moved above the kite''}
	\item \uncover<2->{``\textbf{Peter} moved above the dog and the kite''}
	\item \uncover<3->{78 ppts (after cleaning)}
\end{itemize}

\end{frame}


\begin{frame}{Experiment 2}
	\begin{backgroundblock}{5mm}{15mm}
		\begin{tikzpicture}[framed,background rectangle/.style={thick, rounded corners, draw=black}]
			\node[inner sep=1pt] (screen) at (10,-5)
			{\includegraphics[scale=.55]{gfx/method/build/screen-3}};   		
			\node[font = \scriptsize, below = 0cm of screen] {Stimulus array: A and B moved above C};
		\end{tikzpicture}
	\end{backgroundblock}
	
	\begin{backgroundblock}{65mm}{15mm}
		\begin{tikzpicture}[framed,background rectangle/.style={thick, rounded corners, draw=black}]
			\node[inner sep=3.5pt] (screen) at (10,-5)
			{\includegraphics[scale=.55]{gfx/method/build/screen-4}};   			    					\node[font = \scriptsize, below = 0cm of screen] {Stimulus array: A moved above B and C};
		\end{tikzpicture}
	\end{backgroundblock}
	
	\vspace{3cm}
	
	\begin{small}
	\begin{itemize}
		\item \uncover<1->{Visual confound: syntax planning vs visual encoding}
		\item \uncover<2->{``Peter, the dog, the kite''}
		\item \uncover<3->{If effect isn't syntactic: same slowdown as for conjoined NPs.}
		\item \uncover<4->{No difference if slowdown is syntax-related \parencite{martin2010planning}.}	
		\item \uncover<5->{57 ppts (after cleaning)}	
	\end{itemize}
	\end{small}
\end{frame}